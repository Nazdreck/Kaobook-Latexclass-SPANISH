%----------------------------------------------------------------------------------------
%----------------------------------------------------------------------------------------
%-------------------------------------- PREAMBULO ---------------------------------------
%----------------------------------------------------------------------------------------
%----------------------------------------------------------------------------------------
\documentclass[				% Define la clase del documento y las opciones
	fontsize=10pt, 		 	% Tamaño de Fuente Base.
	twoside=true, 		 	% Si es true, usa diseños distintos para paginas pares e impares.
	%open=any, 			 	% Descomentar para que los capitulos puedan comenzar en cualquier pagina y no solo las pares.
	%chapterprefix=true, 	% Descomentar para usar la palabra "Capitulo" antes de cada numero de capitulo, donde sea que aparezcan
	%chapterentrydots=true, % Descomentar para poner puntos desde el nombre del capitulo al numero de pagina en la tabla de contenidos 
	numbers=noenddot, 		% Elimina el punto detras del numero de cada capitulo, poner enddot para ponerlos devuelta
	%draft=true, 			% Descomentar para agregar reglas al encabezado y pie de pagina
	%overfullrule=true, 	% If uncommented, overly long lines will be marked by a black box; useful for correcting spacing problems
]{kaobook}					% La clase a usar es Kaobook

% Configura el idioma
\usepackage[spanish,es-tabla]{babel} 			% Elige el idioma
\usepackage[spanish=spanish]{csquotes}	% Citas españolas

%----------------------------------------------------------------------------------------
%	ELEMENTOS DEL MAINMATTER
%----------------------------------------------------------------------------------------

% Paquete para realizar testeos:
\usepackage{blindtext}

% Carga paquetes de matematica para teoremas y ambientes relacionados: elegir solo uno entre 'mdftheorems' and 'plaintheorems'.
\usepackage{styles/mdftheorems}			% Ambientes coloreados
%\usepackage{styles/plaintheorems}		% Ambientes sin colores

\graphicspath{{images/}{./}} % Define la ruta donde se guardaran las imagenes

%----------------------------------------------------------------------------------------
%	ELEMENTOS DEL FRONTMATTER Y EL BACKMATTER
%----------------------------------------------------------------------------------------

% Elementos del Indice
\makeindex[columns=3, title=Alphabetical Index, intoc] % Hace que LaTeX produzca los archivos necesarios para compilar el indice

% Elementos de la biblografia
\usepackage{styles/kaobiblio}	% Carga el paquete para la biblografia de la Clase Kaobook
\addbibresource{references.bib} % Define el archivo donde se guardan las referencias bibliograficas

% Elementos del Glosario
\makeglossaries % Hace que LaTeX produzca los archivos necesarios para compilar el glosario

% Elementos de la Nomenclatura
\makenomenclature % Hace que LaTeX produzca los archivos necesarios para compilar la nomenclatura

%----------------------------------------------------------------------------------------
%----------------------------------------------------------------------------------------
%---------------------------------------- CUERPO ----------------------------------------
%----------------------------------------------------------------------------------------
%----------------------------------------------------------------------------------------
\begin{document} % Comienza el contenido del documento

%----------------------------------------------------------------------------------------
%	CARATULA
%----------------------------------------------------------------------------------------

\titlehead{Plantilla de \LaTeX} 	% Aqui va un encabezado del titulo
\subject{Para hacer lindos libros}	% Aqui va un asunto

% El titulo del libro
\title[Plantilla la Clase {\normalfont\texttt{kaobook}}]{Plantilla la Clase {\normalfont\texttt{kaobook}}}
% El subtitulo del libro
\subtitle{Ahora en Español}
% Autor
\author[JBG]{Nahuel E. Gómez}
% Fecha de Publicacion
\date{\today}
% Editorial
\publishers{Un Editor Impresionante}

%----------------------------------------------------------------------------------------
%------------------------------------- FRONTMATTER --------------------------------------
%----------------------------------------------------------------------------------------
\frontmatter % Indica el comienzo del contenido previo al documento, se usan numeros romanos en la tabla de contenidos

%----------------------------------------------------------------------------------------
%	PAGINA DE COPYRIGHT
%----------------------------------------------------------------------------------------
% Editar esta seccion a combeniencia.
\makeatletter
\uppertitleback{\@titlehead} % ENCABEZADO

\lowertitleback{
	\textbf{Disclaimer} \\
	You can edit this page to suit your needs. For instance, here we have a no copyright statement, a colophon and some other information. This page is based on the corresponding page of Ken Arroyo Ohori's thesis, with minimal changes.
	
	\medskip
	
	\textbf{No copyright} \\
	\cczero\ This book is released into the public domain using the CC0 code. To the extent possible under law, I waive all copyright and related or neighbouring rights to this work.
	
	To view a copy of the CC0 code, visit: \\\url{http://creativecommons.org/publicdomain/zero/1.0/}
	
	\medskip
	
	\textbf{Colophon} \\
	This document was typeset with the help of \href{https://sourceforge.net/projects/koma-script/}{\KOMAScript} and \href{https://www.latex-project.org/}{\LaTeX} using the \href{https://github.com/fmarotta/kaobook/}{kaobook} class.
	
	\medskip
	
	\textbf{Publisher} \\
	First printed in May 2019 by \@publishers
}
\makeatother

%----------------------------------------------------------------------------------------
%	DEDICACION
%----------------------------------------------------------------------------------------
% Editar esta seccion a combeniencia.
\dedication{
	The harmony of the world is made manifest in Form and Number, and the heart and soul and all the poetry of Natural Philosophy are embodied in the concept of mathematical beauty.\\
	\flushright -- D'Arcy Wentworth Thompson
}

%----------------------------------------------------------------------------------------
%	IMPRIME LA CARATULA, ENCABEZADO Y DEDICACION
%----------------------------------------------------------------------------------------
% Notar que \maketitle imprime las paginas antes de aqui

% Si twoside=false, \uppertitleback y \lowertitleback no se imprimen
% Para solucionar esto, configuramos twoside=semi justo antes de las paginas de inicio, y volvemos a poner false despues de ellas
\KOMAoptions{twoside=semi}
\maketitle
\KOMAoptions{twoside=false}

%----------------------------------------------------------------------------------------
%	PREFACIO
%----------------------------------------------------------------------------------------

\chapter*{Prefacio}
\blindtext

%----------------------------------------------------------------------------------------
%	TABLA DE CONTENIDOS Y LISTA DE FIGURAS/TABLAS
%----------------------------------------------------------------------------------------

\begingroup % Alcance local para los siguientes comandos

% Define el estilo para TOC, LOF y LOT.

%\setstretch{1} 				% Descomentar para modificar el espacio entre lineas en la TOC
%\hypersetup{linkcolor=blue} 	% Descomenter para configurar el color de los vinculos en la TOC
\setlength{\textheight}{23cm} 	% Ajusta manualmente la altura de las paginas de la TOC

% Activa el modo de compatibilidad para el paquete etoc
\etocstandarddisplaystyle 	% "toc display" como si etoc no estuviera cargado
\etocstandardlines 			% "toc lines" como si etoc no estuviera cargado

\tableofcontents 	% Imprime la Tabla de Contenidos
\listoffigures 		% Imprime la Lista de Figuras

% Comentar las dos siguientes lineas para tenes la LOF y la LOT en paginas diferentes
\let\cleardoublepage\bigskip
\let\clearpage\bigskip

\listoftables 		% Imprime la Lista de Tablasa

\endgroup

%----------------------------------------------------------------------------------------
%------------------------------------- MAINMATTER ---------------------------------------
%----------------------------------------------------------------------------------------
\mainmatter % Indica el comienzo del contenido principal del documento, resetea el conteo de paginas y usa numeros arabigos en la tabla de contenidos

%----------------------------------------------------------------------------------------
%	CAPITULOS
%----------------------------------------------------------------------------------------

% Ejemplo de Capitulo
\setchapterstyle{kao} % Eligue el estilo de encabezado por defecto del capitulo
\chapter{First Chapter}
\blindtext

% Ejemplo de Parte
\pagelayout{wide} 	% Sin margenes
\addpart{Title of the Part}
\pagelayout{margin} % Restaura los margenes

\chapter{Second Chapter}
\blindtext

%----------------------------------------------------------------------------------------
%	APENDICES
%----------------------------------------------------------------------------------------
\appendix % De aqui en adelante los capitulos se numeran con letras latinas

\pagelayout{wide} 	% Sin margenes
\addpart{Apéndices}
\pagelayout{margin} % Restaura los margenes

\chapter{Some more blindtext}
\blindtext

%----------------------------------------------------------------------------------------
%------------------------------------- BACKMATTER ---------------------------------------
%----------------------------------------------------------------------------------------
\backmatter % Indica el fin del contenido princial del cuerpo
\setchapterstyle{plain} % Imprime capitulos planos a partir de este punto

%----------------------------------------------------------------------------------------
%	BIBLIOGRAFIA
%----------------------------------------------------------------------------------------
% La bibliografia necesita ser compilada con biber usando tu editor de LaTeX, o en la linea de comandos con 'biber main' desde el directorio de la plantilla

\defbibnote{bibnote}{En orden de citación.\par\bigskip} % Imprime este texto antes de la bibliografia
\printbibliography[heading=bibintoc, title=Referencias, prenote=bibnote] % Imprime la bibliografia. Ademas, agrega la bibliografia al encabezado de la TOC, configura el titulo de la bibliografia e imprime la nota de la bibliografia definida en la linea anterior

%----------------------------------------------------------------------------------------
%	NOMENCLATURA
%----------------------------------------------------------------------------------------
% La nomenclatura necesita ser compilada en la linea de comandos con 'makeindex main.nlo -s nomencl.ist -o main.nls' desde el directorio de la plantilla

\nomenclature{$c$}{Speed of light in a vacuum inertial frame}
\nomenclature{$h$}{Planck constant}

\renewcommand{\nomname}{Notación} % Configura el nombre de la nomenclatura
\renewcommand{\nompreamble}{La siguiente lista describe los diversos simbolos que se utilizan dentro del libro.} % Antepone este texto a la nomenclatura

\printnomenclature % Imprime la nomenclatura

%----------------------------------------------------------------------------------------
%	ALFABETO GRIEGO
%----------------------------------------------------------------------------------------

\vspace{1cm}
{\usekomafont{chapter}Alfabeto Griego} \\[2ex]
\begin{center}
	\newcommand{\pronounced}[1]{\hspace*{.2em}\small\textit{#1}}
	\begin{tabular}{l l @{\hspace*{3em}} l l}
		\toprule
		Letra & Nombre & Letra & Nombre \\ 
		\midrule
		$\alpha$ & alfa & $\nu$ & nu  \\
		$\beta$ & beta & $\xi$, $\Xi$ & xi \\ 
		$\gamma$, $\Gamma$ & gamma  & o & ómicron \\
		$\delta$, $\Delta$ & delta  & $\pi$, $\Pi$ & pi  \\
		$\epsilon$ & épsilon & $\rho$ & rho  \\
		$\zeta$ & zeta \ & $\sigma$, $\Sigma$ & sigma  \\
		$\eta$ & eta  & $\tau$ & tau  \\
		$\theta$, $\Theta$ & theta  & $\upsilon$, $\Upsilon$ & úpsilon  \\
		$\iota$ & iota  & $\phi$, $\Phi$ & phi  \\
		$\kappa$ & kappa  & $\chi$ & chi  \\
		$\lambda$, $\Lambda$ & lambda  & $\psi$, $\Psi$ & psi  \\
		$\mu$ & mu  & $\omega$, $\Omega$ & omega  \\
		\bottomrule
	\end{tabular} \\[1.5ex]
	Solo se muestran las letras griegas mayúsculas que difieren de las latinas.
\end{center}

%----------------------------------------------------------------------------------------
%	GLOSARIO
%----------------------------------------------------------------------------------------
% El glosario necesita ser compilado en la linea de comandos con 'makeglossaries main' desde el directorio de la plantilla

\newglossaryentry{computer}{
	name=computer,
	description={is a programmable machine that receives input, stores and manipulates data, and provides output in a useful format}
}

% Glossary entries (used in text with e.g. \acrfull{fpsLabel} or \acrshort{fpsLabel})
\newacronym[longplural={Frames per Second}]{fpsLabel}{FPS}{Frame per Second}
\newacronym[longplural={Tables of Contents}]{tocLabel}{TOC}{Table of Contents}

\setglossarystyle{listgroup} % Configura el estilo del glosario (ver https://en.wikibooks.org/wiki/LaTeX/Glossary para una referencia)
\printglossary[title=Glosario, toctitle=Glosario] % Imprime el glosario, 'title' es el encabezado del glosario, toctitle es el encabezado que se muestra en la toc

%----------------------------------------------------------------------------------------
%	INDEX
%----------------------------------------------------------------------------------------
% El indice necesita ser compilado en la linea de comandos con 'makeindex main' desde el directorio de la plantilla

\printindex % Imprime el indice

%----------------------------------------------------------------------------------------
%	CONTRAPORTADA
%----------------------------------------------------------------------------------------
% Si tenes una imagen/archivo PDF que quieras usar como contraportada, descomenta las siguientes lineas.

%\clearpage
%\thispagestyle{empty}
%\null%
%\clearpage
%\includepdf{cover-back.pdf}

%----------------------------------------------------------------------------------------
\end{document}
